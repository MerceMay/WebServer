\noindent
\textbf{高性能 Web 服务器(C++)} \hfill 2025.04 – 2025.05 \\
\textbf{项目描述}:基于 Reactor 模式从零实现了一个高并发、事件驱动的静态 Web 服务器,支持完整的 HTTP/1.1 协议解析与响应生成,压力测试达到 QPS 15,000+,具备良好的可扩展性和稳定性。
\begin{itemize}
    \item \textbf{主从 Reactor 架构设计}:采用主从 Reactor 多线程模式,主线程 EventLoop 专注于监听 socket 的连接接受,通过 \texttt{accept()} 获取新连接后,利用轮询策略将连接分发至子线程的 EventLoop 进行 I/O 处理,实现了连接管理与 I/O 处理的职责分离,提升系统并发能力。
    
    \item \textbf{事件驱动与线程池}:每个子线程封装独立的 \texttt{epoll} 实例和 \texttt{eventfd}。主线程接受连接后,通过任务队列(\texttt{queueInLoop})安全地将 \texttt{HttpData} 初始化任务提交到目标子线程,并利用 \texttt{eventfd} 触发子线程唤醒机制,确保跨线程任务调度的线程安全性和低延迟唤醒。
    
    \item \textbf{HTTP 协议状态机解析器}:设计并实现基于有限状态机的 HTTP 请求解析器,支持请求行(Request Line)、请求头(Headers)、请求体(Body)的分阶段解析。状态机能够处理 TCP 数据包的不完整到达情况,当数据不足时返回 \texttt{AGAIN} 状态挂起解析流程,待数据就绪后从断点恢复解析,保证协议解析的鲁棒性和容错性。
    
    \item \textbf{异步日志系统}:构建了基于双缓冲区和生产者-消费者模型的异步日志系统。前端(业务线程)负责日志格式化并写入当前缓冲区,后端(日志线程)定期将满载的缓冲区交换并刷盘,实现日志生成与磁盘 I/O 的完全解耦,避免阻塞业务逻辑,显著降低日志写入对性能的影响。
    
    \item \textbf{定时器与连接管理}:基于最小堆(\texttt{priority\_queue})实现高效定时器管理,自动监控并清理超时的空闲连接。定时器与 \texttt{HttpData} 通过 \texttt{weak\_ptr} 松耦合关联,防止循环引用导致的内存泄漏,及时释放系统资源,提升服务器长时间运行的稳定性。
    
    \item \textbf{技术特性}:采用 \texttt{epoll} 边缘触发模式(ET)实现高效 I/O 多路复用;所有 socket 设置为非阻塞模式;使用 RAII 机制管理资源生命周期;通过互斥量(\texttt{std::mutex})和条件变量(\texttt{std::condition\_variable})保证线程间的同步与通信;支持 HTTP 持久连接(Keep-Alive)以减少连接开销。
\end{itemize}
\textbf{技术栈}:C++20、epoll、主从 Reactor 模式、线程池、状态机、互斥量、条件变量、RAII、非阻塞 I/O、eventfd、最小堆、智能指针
\vspace{0.2cm}
